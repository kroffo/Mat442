\documentclass{scrartcl}
\usepackage{amsmath,amssymb,commath}
\setkomafont{disposition}{\normalfont\bfseries}

\title{Complex Analysis}
\subtitle{Homework 1: 1.1) 16, 30, 42}
\author{Kenny Roffo}
\date{Due August 26, 2015}

\begin{document}
\maketitle
\textbf{16)} Write the number $\frac{(4+5i)+2i^3}{(2+i)^2}$ in the form $a+bi$.\\

\begin{align*}
  \frac{(4+5i)+2i^3}{(2+i)^2} &= \frac{(4 + 5i) - 2i}{4 + 4i - 1}\\
  &= \frac{4+3i}{3+4i}\\
  &= \frac{(4+3i)(3-4i)}{(3+4i)(3-4i)}\\
  &= \frac{12-16i+9i+12}{9+16}\\
  &= \frac{24-7i}{25}\\
  &= \frac{24}{25}-\frac{7}{25}i
\end{align*}\\

\textbf{30)} Let $z=x+yi$. Express $\text{Im}(\bar{z}^2 + z^2)$ in terms of $x$ and $y$.

\begin{align*}
  \bar{z}^2 + z^2 &= (x-yi)^2 + (x+yi)^2\\
  &= x^2 - 2xyi - y^2 + x^2 + 2xyi - y^2\\
  &= 2(x^2 - y^2) + 0i\\
  \implies &\text{Im}(\bar{z}^2 + z^2) = 0
\end{align*}\\

\pagebreak
\textbf{42)} Solve the equation $\frac{z}{1+\bar{z}}=3+4i$ for $z=a+bi$.

Note that $z=a+bi$ and $\bar{z}=a-bi$. Plugging in, we have
\begin{align*}
  \frac{a+bi}{1+a-bi} &= 3+4i\\
  \implies a+bi &= (3+4i)((1+a)-bi)\\
  &= 3+3a-3bi+4i+4ai+4b\\
  &= (3a+4b+3)+(4a-3b+4)i
\end{align*}
By the definition of equality of complex numbers, we see that the above equality implies

$$a=3a+4b+3 \text{\hspace{0.4in} and \hspace{0.4in}} b=4a-3b+4$$

We see the first equation yields the value of $b$ in terms of $a$:
\begin{align*}
  &b = 4a-3b+4\\
  \implies &4b = 4a+4\\
  \implies &b = a+1
\end{align*}

and plugging in for $b$ in the first equation we can solve for $a$:
\begin{align*}
  &a = 3a+4b+3 = 3a+(4a+4)+3\\
  \implies &-6a = 7\\
  \implies &a = -\frac{6}{7}
\end{align*}

Thus, simply plugging in the value of $a$ to get $b$ we have

$$z=-\frac{6}{7}-\frac{1}{6}i$$

\end{document}
