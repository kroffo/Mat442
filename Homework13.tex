\documentclass{scrartcl}
\usepackage{amsmath,amssymb,commath}
\setkomafont{disposition}{\normalfont\bfseries}

\title{Complex Analysis}
\subtitle{Homework 13: 3.3) 24, 40}
\author{Kenny Roffo}
\date{Due October 19, 2015}

\begin{document}
\maketitle

\textbf{24 a)} Show the function $f(z) = x^2 - x + y + i(y^2 - 5y - x)$ is not analytic at any point but is differentiable along the curve $y=x+2$\\

We see $u(x,y) = x^2 - x + y$ and $v(x,y) = y^2 - 5y - x$. So
\begin{align*}
  u_x &= 2x - 1\\
  u_y &= 1\\
  v_x &= -1\\
  v_y &= 2y - 5
\end{align*}

In order for the function to be differentiable at a point $z$, the C-R equations must hold. It is true for all points in the complex plane that $u_y = -v_x$, but let us examine the second C-R equation, $u_x = v_y$.
\begin{align*}
  &2x - 1 = 2y - 5\\
  \iff& 2x + 4 = 2y\\
  \iff& x + 2 = y
\end{align*}
Therefore, the C-R equations hold at a point $z$ if and only if $z$ is on the line $y = x + 2$. Thus, $f$ is not differentiable (and thus not analytic) at any point not on the line $y=x+2$. Also, since every neighborhood of every point on the line $y = x + 2$ contains a point not on the line $y=x+2$, every neighborhood of every point on the line contains a point which is not differentiable, thus $f$ is not analytic at any point on the line $y=x+2$. Therefore, $f$ is nowhere analytic.

Note that $u$, $v$, $u_x$, $u_y$, $v_x$, and $v_y$ are all polynomial (or constant) functions. Then all $u$ and $v$ and their first order partial derivatives are all continuous. Since this is true, and the C-R equations hold for points on the line $y=x+2$, the function $f$ is differentiable on the line $y=x+2$.
\pagebreak

\textbf{24 b)} Find the derivative of $f(z) = x^2 - x + y + i(y^2 - 5y - x)$ on the curve $y=x+2$.\\
Using the fact that $f'(z) = \pd{u}{z}+i\pd{v}{x}$ we find the derivative:

\begin{align*}
  f'(z) &= \pd{u}{z}+i\pd{v}{x}\\
  &= 2x-1 + i(-1)\\
  &= 2x - 1 - i
\end{align*}\\

\textbf{40)} Consider the function 
$$f(z) = 
\begin{cases} 
  0 & z = 0 \\
  \frac{z^5}{|z^4|} & z\ne0
\end{cases}$$

\textbf{   a)} Express $f$ in the form $f(z) = u(x,y) + iv(x,y)$\\

We first express $\frac{z^5}{|z^4|}$ in this form:

\begin{align*}
  \frac{z^5}{|z^4|} &= \frac{(x+iy)^5}{|(x+iy)^4|}\\
  &= \frac{x^5 + 5x^4iy + 10x^3i^2y^2 + 10x^2i^3y^3 + 5xi^4y^4 + i^5y^5}{|x^4 + 4x^3iy + 6x^2i^2y^2 + 4xi^3y^3 + i^4y^4|}\\
  &= \frac{(x^5 - 10x^3y^2 + 5xy^4)+i(5x^4y - 10x^2y^3 + y^5)}{|(x^4-6x^2y^2+y^4)+i(4x^3y - 4xy^3)|}\\
  &= \frac{(x^5 - 10x^3y^2 + 5xy^4)+i(5x^4y - 10x^2y^3 + y^5)}{\sqrt{(x^4-6x^2y^2+y^4)^2+(4x^3y - 4xy^3)^2}}\\
  &= \frac{(x^5 - 10x^3y^2 + 5xy^4)+i(5x^4y - 10x^2y^3 + y^5)}{\sqrt{x^8+4x^6y^2 + 6x^4y^4 + 4x^2y^6 + y^8}} \text{\hspace{0.5in}\small{After a bit of algebra}}\\
  &= \frac{(x^5 - 10x^3y^2 + 5xy^4)+i(5x^4y - 10x^2y^3 + y^5)}{\sqrt{(x^2+y^2)^4}}\\
  &= \frac{(x^5 - 10x^3y^2 + 5xy^4)+i(5x^4y - 10x^2y^3 + y^5)}{(x^2+y^2)^2}\\
  &= \frac{(x^5 - 10x^3y^2 + 5xy^4)}{(x^2+y^2)^2}+i\frac{(5x^4y - 10x^2y^3 + y^5)}{(x^2+y^2)^2}
\end{align*}\\

Now we have $f(z)$ given by
\begin{displaymath}
  f(z) = 
  \begin{cases} 
    0 & z = 0 \\
    \left(\frac{x^5 - 10x^3y^2 + 5xy^4}{{(x^2+y^2)^2}}\right) + i\left(\frac{5x^4y - 10x^2y^3 + y^5}{{(x^2+y^2)^2}}\right) & z\ne0
  \end{cases}
\end{displaymath}
\pagebreak

\textbf{   b)} Show that $f$ is not differentiable at the origin.\\

Consider the limit as $f$ approaches $0$ along the imaginary axis ($y=0$):

\begin{align*}
\lim_{x\rightarrow0}\frac{(x+i(0))^5}{|(x+i(0))^4|} &= \lim_{x\rightarrow0}\frac{x^5}{|x^4|}\\
  &= \lim_{x\rightarrow0}\frac{x^5}{x^4}\\
  &= \lim_{x\rightarrow0}x\\
  &= 0
\end{align*}

Now consider the limit as $f$ approaches $0$ along the line $y=x+1$:

\begin{align*}
  \lim_{x\rightarrow0}\frac{(x+i(x+1))^5}{|(x+i(x+1))^4|} &= \lim_{x\rightarrow0}\frac{(0+i(0+1))^5}{|(0+i(0+1))^4|}\\
  &= \lim_{x\rightarrow0}\frac{i^5}{|i^4|}\\
  &= \lim_{x\rightarrow0}\frac{i^5}{1}\\
  &= i^5
\end{align*}

So we get different values for the limit depending on which path of approach we choose, thus the limit does not exist. This means that the $f$ is discontinuous at 0, thus $f$ is not differentiable at 0.\pagebreak

\textbf{   c)} Show that the Cauchy-Riemann equations are satisfied at the origin.\\

We begin by finding the partial derivatives $u_x, u_y, v_x$ and $v_y$ at the point 0. At 0, $u = v = 0$. Using the limit definition for partial derivatives, we see

\begin{align*}
  u_x &= \lim_{\Delta x\rightarrow0}\frac{u(x+\Delta x,y) - u(x,y)}{\Delta x}\\
  &= \lim_{\Delta x\rightarrow0}\frac{0 - 0}{\Delta x}\\
  &= \lim_{\Delta x\rightarrow0}0\\
  &= 0
\end{align*}

\begin{align*}
  u_y &= \lim_{\Delta y\rightarrow0}\frac{u(x,y+\Delta y) - u(x,y)}{\Delta y}\\
  &= \lim_{\Delta y\rightarrow0}\frac{0 - 0}{\Delta y}\\
  &= \lim_{\Delta y\rightarrow0}0\\
  &= 0
\end{align*}

\begin{align*}
  v_x &= \lim_{\Delta x\rightarrow0}\frac{v(x+\Delta x,y) - v(x,y)}{\Delta x}\\
  &= \lim_{\Delta x\rightarrow0}\frac{0 - 0}{\Delta x}\\
  &= \lim_{\Delta x\rightarrow0}0\\
  &= 0
\end{align*}

\begin{align*}
 v_y &= \lim_{\Delta y\rightarrow0}\frac{v(x,y+\Delta y) - v(x,y)}{\Delta y}\\
  &= \lim_{\Delta y\rightarrow0}\frac{0 - 0}{\Delta y}\\
  &= \lim_{\Delta y\rightarrow0}0\\
  &= 0
\end{align*}

Since 0 = 0 = -0, the Cauchy-Riemann equations are satisfied for $f$ at $z=0$.
\end{document}
