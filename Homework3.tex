\documentclass{scrartcl}
\usepackage{amsmath,amssymb,commath}
\setkomafont{disposition}{\normalfont\bfseries}

\title{Complex Analysis}
\subtitle{Homework 3: 1.3) 34, 44 \\ 1.4) 18}
\author{Kenny Roffo}
\date{Due September 2, 2015}

\begin{document}
\maketitle

\textbf{34)} Use de Moivre's formula with $n=3$ to find trigonometric identities for $\cos{3\theta}$ and $\sin{3\theta}$.\\

De Movire's formula says $\left(\cos{\theta} + i\sin{\theta}\right)^n = \cos{n\theta} + i\sin{n\theta}$. Applying $n=3$ to the formula, we see
\begin{align*}
  \cos{3\theta} + i\sin{3\theta} &= \left(\cos{\theta} + i\sin{\theta}\right)^3\\
  &= \left(\cos^2{\theta}-\sin^2{\theta}+2i\cos{\theta}\sin{\theta}\right)\left(\cos{\theta}+i\sin{\theta}\right)\\
  &= \cos^3{\theta} - 3\sin^2{\theta}\cos{\theta} + i\left(-\sin^3{\theta} + 3\cos^2{\theta}\sin{\theta}\right)
\end{align*}

By the definition of equality of complex numbers, this implies
$$ \cos{3\theta} = \cos^3{\theta} - 3\sin^2{\theta}\cos{\theta} \text{\hspace{1in} and \hspace{1in}} \sin{3\theta} = -\sin^3{\theta} + 3\cos^2{\theta}\sin{\theta} $$\\

\textbf{44)} Describe the set of points $z$ in the complex plane that satisfy $\text{arg}(z)=\frac{\pi}{4}$.\\

All $z \in \mathbb{C}$ can be written in polar form as $r\left(\cos{\theta} + i\sin{\theta}\right)$. In our particular case, we know $\theta = \frac{\pi}{4}$, so we must describe the set of all $z$ of the form $$r\left(\cos{\frac{\pi}{4}} + i\sin{\frac{\pi}{4}}\right) = r\frac{\sqrt{2}}{2} + ir\frac{\sqrt{2}}{2}$$. This tells us that the set of interest is the set of all $z \in \mathbb{C}$ such that the real and imaginary parts are equal.\pagebreak

\textbf{18)} Use the fact that $8i = (2 + 2i)^2$ to find all solutions of the equation $z^2 -8z + 16 = 8i$.\\

To solve this problem, we will simply find all the $2^{nd}$ roots of $(2 + 2i)^2$. First we must express the complex number $(2 + 2i)$ in polar form. We see the radius is
\begin{align*}
r &= \sqrt{2^2 + 2^2}\\
  &= 2\sqrt{2}
\end{align*}
and the angle is found using basic trigonometry:
\begin{align*}
  \theta &= \tan^{-1}{\frac{2}{2}}\\
        &= \frac{\pi}{4}
\end{align*}
Therefore we can write $(2 + 2i) = 2\sqrt{2}\left(\cos{\frac{\pi}{4}} + i\sin{\frac{\pi}{4}}\right)$. Now we know from how powers of complex numbers work
$$ (2 + 2i)^2 = 8\left(\cos{\frac{\pi}{2}} + i\sin{\frac{\pi}{2}}\right) $$
Finally, we can use the formula $$\phi = \frac{\theta + 2k\pi}{n}$$ with values of $k$ from 0 up to $n-1$ (in this case 1) to find all distinct $2^{nd}$ roots of $(2 + 2i)^2$ by the formula $w_k = \sqrt{r}\left(\cos{\phi}+i\sin{\phi}\right)$.
\begin{align*}
  k=0 : w_0 &= \sqrt{8}\left(\cos\left(\frac{\frac{\pi}{2}+2(0)\pi}{2}\right)+i\sin\left(\frac{\frac{\pi}{2}+2(0)\pi}{2}\right)\right)\\
            &= \sqrt{8}\left(\cos\left(\frac{\pi}{4}\right)+i\sin\left(\frac{\pi}{4}\right)\right)\\
  k=1 : w_1 &= \sqrt{8}\left(\cos\left(\frac{\frac{\pi}{2}+2(1)\pi}{2}\right)+i\sin\left(\frac{\frac{\pi}{2}+2(1)\pi}{2}\right)\right)\\
            &= \sqrt{8}\left(\cos\left(\frac{5\pi}{4}\right)+i\sin\left(\frac{5\pi}{4}\right)\right)
\end{align*}

Looking back at the original problem, we realize that $z^2 -8z + 16$ is really just $(z - 4)^2$, so if we add $4$ to each of our roots of $8i$ we will have our $z$'s, thus the solutions to the equation are
$$ z=\sqrt{8}\left(\cos\left(\frac{\pi}{4}\right)+i\sin\left(\frac{\pi}{4}\right)\right) + 4 \text{\hspace{0.5 in} and \hspace{0.5 in}} z=\sqrt{8}\left(\cos\left(\frac{5\pi}{4}\right)+i\sin\left(\frac{5\pi}{4}\right)\right) + 4 $$
\end{document}
